	\chapter{Einleitung}
	\section{Motivation}
	TODO: astrophysikalischer Kontext $\rightarrow$ Begründung der in Sektion (\ref{sec:radiative_transfer}) gemachten Annahmen
	\section{Ziele der Arbeit}
	In dieser Arbeit wird eine Pfadintegraldarstellung der Strahlungstransportgleichung hergeleitet, sowie dessen Nutzen für Monte--Carlo--Verfahren zur Lösung des Strahlungstransportproblems theoretisch und praktisch in Form eines Computerprogrammes gezeigt.
	
	\section{Nomenklatur}\label{subsec:nomenklatur}
	Im folgenden Text werden die in Tabelle \ref{tab:nomenklatur} angegebenen Schreibweisen benutzt.

	\begin{table}
		\caption{Nomenklatur}
		\begin{center}
		\begin{tabular}{rll}
			Schreibweise & Bedeutung & Einheit \\
			\hline
			$\kappa$ & Volumenabsorptionsquerschnitt & $\left[\text{m}^2/\text{m}^3\right]$ \\
			$\sigma$ & Volumenstreuquerschnitt & $\left[\text{m}^2/\text{m}^3\right]$ \\
			$\xi$ & Volumenextinktionsquerschnitt & $\left[\text{m}^2/\text{m}^3\right]$ \\
			$\varepsilon$ & Volumenemissivität & $\left[\text{W}/(\text{m}^3\,\text{sr})\right]$ \\
			$I(\location{r},\omega)$ & Intensität am Ort $\location{r}$ in Richtung $\omega$& $\left[\text{W}/(\text{m}^2\,\text{sr})\right]$ \\
			$W(\location{r},\omega)$ & Sensitivität am Ort $\location{r}$ in Richtung $\omega$ & $\left[(\text{m}^2\,\text{sr})/\text{W}\right]$ \\
			$k(\location{r},\omega',\omega)$ & Phasenfunktion am Ort $\location{r}$ für ein Teilchen, & $\left[1/\text{sr}\right]$ \\
				&das sich vor der Streuung in Richtung $\omega'$&\\
				&und nach der Streuung in Richtung $\omega$ bewegt& \\
			$\scatter{i}$ & Produkt aus Volumenstreuquerschnitt und&\\
			  & Streuphasenfunktion am Ort $\location{r}_i$ für ein aus Richtung $\location{r}_{i-1}$&\\ 
				&kommendes und in Richtung $\location{r}_{i+1}$ gestreutes Teilchen&\\
				&(äquivalent zu $\sigma(\location{r}_i)k(\location{r}_i,\normalized{\location{r}_i-\location{r}_{i-1}},\normalized{\location{r}_{i+1}-\location{r}_i})$)& \\
			$\tau_{(i,j)}$ & Optische Tiefe zwischen $\location{r}_i$ und $\location{r}_j$ & \\
			$\varepsilon_{(i,j)}$ & Emissivität am Ort $\location{r}_i$ in Richtung $\location{r}_j$ & \\
			$W_{(i,j)}$ & Sensitivität am Ort $\location{r}_j$ für Strahlung in Richtung $\normalized{\location{r}_j-\location{r}_i}$ &
		\end{tabular}
		\end{center}
		\label{tab:nomenklatur}
	\end{table}
