\chapter{Zusammenfassung und Ausblick}
	Wir haben in dieser Arbeit eine alternative Lösung des Strahlungstransportproblems vorgestellt, in der beliebige Messungen des Strahlungsfeldes, das die Strahlungstransportgleichung löst, als Integral über den Raum aller Photonenpfade formuliert sind. Wir haben gesehen, dass Monte--Carlo--Integration gut für die Berechnung dieses Integrals geeignet ist, da ihre Konvergenzrate im Gegensatz zu klassischen numerischen Quadraturverfahren nicht von der Dimension des Integrationsgebietes abhängt. Im Unterschied zu anderen Monte--Carlo--Verfahren zur Lösung des Strahlungstransportproblems behandelt die vorgestellte Lösung immer komplette Photonenpfade und ist dadurch in der Lage, den Messbeitrag eines Pfades vollständig von seiner Generierungswahrscheinlichkeit zu entkoppeln. Dies erlaubt eine größere Freiheit bei der Konstruktion von Lösungsverfahren die zur Effizienzsteigerung genutzt werden kann.
	Eine Implementation dieser Lösungsmethode wurde mit dem Programm \pirate vorgestellt.
	
	Für den Testfall einer homogenen Kugel aus isotrop streuendem Material mit einer Punktlichtquelle im Zentrum wurde ein numerisch schnell berechenbarer geschlossener Ausdruck hergeleitet, sowie ein Monte--Carlo--Schema vorgestellt, welches schnell gegen die Lösung konvergiert. Als zweiter Testfall haben wir eine inhomogene Dichteverteilung in Form eines einfachen Scheibenmodells betrachtet.

	Anschließend wurden für die Testfälle Simulationsergebnisse, die mittels der Programme \mctd und \pirate berechnet wurden, vorgestellt.
	Mit \pirate wurde demonstriert, dass der Lösungsansatz mittels Pfadintegralen zu einer Effizienzsteigerung um mehrere Größenordnungen führen kann: bei optisch dünnen Systemen verhindert ein Distanz--Sampler, der eine Streuung erzwingt, dass Photonen das System verlassen und nicht in die Messung eingehen. Im Falle inhomogenen Dichteverteilungen garantiert die Freiheit, die Generierung der Pfade beim Sensor starten zu lassen, dass nur Photonen berücksichtigt werden, die den Sensor erreichen.
	
	Für die Weiterentwicklung von \pirate ist es nötig, die genaue Ursache der inkorrekten Schätzungen für den direkten Lichtanteil sowie die Abweichungen im Scheibentestfall herauszufinden und zu korrigieren. Eine Normierung der erzeugten Bilder mittels Monte--Carlo--Integration für absolute Intensitäten statt relativer Intensitäten sollte leicht zu implementieren sein. Darüber hinaus ist die Implementation realistischer Streuphasenfunktionen, verschiedener Wellenlängen zur Berechnung von spektralen Energieverteilungen wichtig, um in der Praxis von Nutzen zu sein. 
	
	In optisch dicken Konfigurationen ($\tau>10$) wird der einfache sensorbasierte Pfadgenerierungsansatz, der in \pirate implementiert ist ineffizient. Daher ist die Konstruktion von Pfadgenerierungsverfahren erstrebenswert, die auch in diesem Fall effizient funktionieren. Ein hierfür vielversprechender Ansatz findet sich in \citep{Laszlo:2005p11056,DAldous:1994p11528,Grassberger:2002p10876}. Auch die Implementation des ``Multiple--Try''--Metropolis--Hastings--Algorithmus \citep{Liu:2000p8427} wäre interessant. Die Freiheit bei der Wahl der genutzten Sensitivitätsfunktion $W$ in der Messgleichung erlaubt ausserdem die Konstruktion von Sensoren die maximalen Gebrauch von Symmetrien im betrachteten Problem machen, indem Sensoren überall dort empfindlich sind, wo eine passend gedrehte Kamera das Photon für das betrachtete Pixel gezählt hätte, wie dies z.B. das in \ref{subsec:homspherefastmcscheme} vorgestellte Schema für kugelsymmetrische Konfigurationen macht.

	