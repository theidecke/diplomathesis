\documentclass[11pt,a4paper,DIVcalc,BCOR8mm,titlepage,twoside]{scrartcl}

%\usepackage[parfill]{parskip}    % Activate to begin paragraphs with an empty line rather than an indent
\usepackage{german}
\usepackage{amsmath}
\usepackage{amsfonts}
\usepackage{amssymb}
\usepackage{graphicx}
\usepackage{natbib}
\bibpunct{(}{)}{;}{a}{}{,} % to follow the A&A style

\begin{document}
	\titlehead{Christian--Albrechts--Universit"at zu Kiel\\ Institut f"ur Theoretische Physik und Astrophysik}
	\subject{Diplomarbeit}
	\title{L"osung des Strahlungstransportproblems in Pfadintegralform mit effizienten Monte--Carlo--Verfahren}
	\author{vorgelegt von Thies Heidecke\\(theidecke@astrophysik.uni-kiel.de)}
	\publishers{betreut durch Prof. Sebastian Wolf}
	\date{Version vom \today}
	\maketitle
	

	\tableofcontents
	\vfill
	\pagebreak
	
	\newcommand{\location}[1]{\mathbf{#1}}
	\newcommand{\scatter}[1]{\overset{#1}{\leftrightsquigarrow}}
	\newcommand{\normalized}[1]{\frac{#1}{||#1||}}
	
	\section{Einleitung}
	\subsection{Motivation}
	\subsection{Ziele der Arbeit}
	In dieser Arbeit soll eine Pfadintegraldarstellung der Strahlungstransportgleichung hergeleitet werden, sowie dessen Nutzen f"ur Monte--Carlo--Verfahren zur L"osung des Strahlungstransportproblems theoretisch und praktisch in Form eines Computerprogrammes gezeigt werden.
	
	\subsection{Nomenklatur}
	Im folgenden Text werden die in Tabelle \ref{tab:nomenklatur} angegebenen Schreibweisen benutzt.

	\begin{table}
		\caption{Nomenklatur}
		\begin{center}
		\begin{tabular}{rll}
			Schreibweise & Bedeutung & Einheit \\
			\hline
			$\kappa$ & Volumenabsorptionsquerschnitt & $\left[\text{m}^2/\text{m}^3\right]$ \\
			$\sigma$ & Volumenstreuquerschnitt & $\left[\text{m}^2/\text{m}^3\right]$ \\
			$\xi$ & Volumenextinktionsquerschnitt & $\left[\text{m}^2/\text{m}^3\right]$ \\
			$\varepsilon$ & Volumenemissivit"at & $\left[\text{W}/(\text{m}^3\,\text{sr})\right]$ \\
			$I(\location{r},\omega)$ & Intensit"at am Ort $\location{r}$ in Richtung $\omega$& $\left[\text{W}/(\text{m}^2\,\text{sr})\right]$ \\
			$W(\location{r},\omega)$ & Sensitivit"at am Ort $\location{r}$ in Richtung $-\omega$ & $\left[(\text{m}^2\,\text{sr})/\text{W}\right]$ \\
			$\phi(\location{r},\omega',\omega)$ & Phasenfunktion am Ort $\location{r}$ f"ur ein Teilchen, & $\left[1/\text{sr}\right]$ \\
				&das sich vor der Streuung in Richtung $\omega'$&\\
				&und nach der Streuung in Richtung $\omega$ bewegt& \\
			$\scatter{i}$ & Phasenfunktion am Ort $\location{r}_i$ f"ur ein aus Richtung $\location{r}_{i-1}$&\\ 
				&kommendes und in Richtung $\location{r}_{i+1}$ gestreutes Teilchen&\\
				&("aquivalent zu $\phi(\location{r}_i,\normalized{\location{r}_i-\location{r}_{i-1}},\normalized{\location{r}_{i+1}-\location{r}_i})$)& \\
			$\tau_{i,j}$ & Optische Tiefe zwischen $\location{r}_i$ und $\location{r}_j$ & \\
			$\varepsilon_{i,j}$ & Emissivit"at am Ort $\location{r}_i$ in Richtung $\location{r}_j$ & \\
			$W_{j,i}$ & Sensitivit"at am Ort $\location{r}_i$ in Richtung $-\normalized{\location{r}_j-\location{r}_i}$ &
		\end{tabular}
		\end{center}
		\label{tab:nomenklatur}
	\end{table}
	
	
	\section{Das Strahlungstransportproblem}
	TODO: Strahlungsgr"o"sen, Me"sgleichung, STG in diff. Form erkl"aren, "Ubersicht benutzter L"osungsverfahren
	
	Das Verhalten von Licht l"asst sich (nach heutigem wissenschaftlichen Stand) durch die {\em Quantenelektrodynamik} in allen Details vollst"andig beschreiben. Es beinhaltet solche Ph"anomene wie Dispersion, Brechung, Interferenz, Photon--Photon--Interaktion, etc. Diese Effekte sind h"aufig dann am bedeutendsten, wenn die Ausma"se der betrachteten Objekte von der Gr"o"senordnung der Wellenl"ange des Lichtes sind. Auf der anderen Seite beschreibt die {\em geometrische Optik} die rein makroskopische lineare Ausbreitung gro"ser Mengen von Photonen ohne obengenannte Wellen--Ph"anomene zu ber"ucksichtigen.
	
	Beim Strahlungstransportproblem (STP) sind wir an einer {\em ph"anomenologischen} Beschreibung interessiert. Das heisst wir wollen die Effekte des Lichtes modellieren, die in typischen Anwendungen durch optische Instrumente (Auge, Teleskope mit Photoplatten/CCD--Chips) gemessen werden k"onnen. Dies bedeutet, das wir haupts"achlich eine geometrische Beschreibung des Lichtes in Form eines Partikeltransportproblems ansetzen aber relevante quantenmechanische Effekte wie Photonen--Streuung in erster Ordnung lokal mitber"ucksichtigen (z.B. in Form einer Streuphasenfunktion).
	
	\subsection{Das Strahlungstransportproblem als Partikeltransportproblem}
	Um Strahlungstransport als Partikeltransportproblem behandeln zu k"onnen m"ussen folgende Bedingungen erf"ullt sein
	\begin{itemize}
		\item{Die Partikel m"ussen so klein und zahlreich sein, das ihre statistische Verteilung als kontinuierlich angesehen werden kann}
		\item{Zu jedem Zeitpunkt l"asst sich ein Partikel komplett durch seinen Positionsvektor, Geschwindigkeitsvektor und eventuelle interne Zust"ande (wie Polarisation, Frequenz, Ladung, Spin, etc.) charakterisieren}
	\end{itemize}
	Diese Annahmen sind f"ur Photonen und die uns interessierenden r"aumlichen Entfernungen erf"ullt.
	Dar"uber hinausgehend machen wir im Rahmen dieser Arbeit folgende Annahmen:
	\begin{itemize}
		\item{Die Materialeigenschaften variieren bei Variation des Ortes in der Gr"o"senordnung der Wellenl"ange nur wenig}
		\item{Das Strahlungsintensit"atsfeld ist station"ar (d.h. innerhalb der typischen Zeiten, die ein Photon braucht um das Simulationsgebiet zu durchqueren, k"onnen die Materialeigenschaften als statisch angenommen werden)}
		\item{Photonen werden ausschliesslich elastisch gestreut}
		\item{der Raum wird als euklidisch angenommen (d.h. es werden keine relativistischen Effekte ber"ucksichtigt)}
	\end{itemize}
	\subsection{Die Me"sgleichung}
	\subsection{Strahlungstransportgleichung in differentieller Form}
		
	\section{Pfadintegralformulierung der Strahlungstransportgleichung}
	TODO: STG in diff. Form, STG in integraler From, STG in Operator--Form, Messungen als Pfadintegrale
	\subsection{Strahlungstransportgleichung in integraler Form}
	TODO: STG in integraler From
	\subsection{Strahlungstransportgleichung in Operator--Form}
	TODO: STG in Operator--Form
	\subsection{KG--Operator in Volumenform}
	\subsection{Pfadintegrall"osung der Strahlungstransportgleichung}
	TODO: Messungen als Pfadintegrale
	\subsection{Strahlungstransport als Integrationsproblem}
	
	\section{Monte--Carlo--Integration}
	\subsection{Grundlegende Begriffe aus der Statistik}
	TODO: Begriffe aus Wahrscheinlichkeitsrechnung und Statistik einf"uhren
	\subsection{Das Integrationsproblem}
	TODO: Das Integrationsproblem
	\subsection{Vergleich von Tensor--Produkt--Verfahren und einfacher Monte--Carlo--Integration}
	TODO: (Curse~of~Dimensionality / Konvergenzraten)
	\subsection{Importance--Sampling}

	\subsection{Monte--Carlo--Markov--Chain--Verfahren}
	\subsubsection{Metropolis--Hastings--Algorithmus}
	TODO: Metropolis--Hastings--Algorithmus
	\subsubsection{Detailed Balance}
	TODO: Detailed Balance
	\subsubsection{Generalisierter Metropolis--Hastings--Algorithmus}
	TODO: Robuste MH--Variante

	\section{Monte--Carlo--Strahlungstransport}
	\subsection{Pfadgenerierung}
	\subsubsection{Raycasting}
	\subsubsection{Distanzsampler}
	TODO: uniform depth sampler, uniform attenuation sampler, enforced uniform attenuation sampler
	
	TODO: Pfadgenerierung, Sch"atzer
	
	\section{Testf"alle}
	\subsection{Homogene streuende Kugel}
	\subsubsection{analytische L"osung}
	\subsection{Einfaches Scheibenmodell}
	
	\section{Programmbeschreibung}
	TODO: Programmbeschreibung in Worten bzw. Pseudocode
	
	\section{Resultate}
	\subsection{Homogen streuende Kugel}
	TODO: Vergleich mit analytischer L"osung und MC3D (Ergebnis+Geschwindigkeit)
	\subsection{Einfaches Scheibenmodell}
	TODO: Vergleich mit MC3D (Ergebnis und Geschwindigkeit)
	
	\section{Zusammenfassung und Ausblick}

	\bibliographystyle{chicago}
	\bibliography{bibliography}
\end{document}