\documentclass[11pt,a4paper]{scrartcl}

%\usepackage[parfill]{parskip}    % Activate to begin paragraphs with an empty line rather than an indent
\usepackage{german}
\usepackage{amsmath}
\usepackage{amsfonts}
%\usepackage{extarrows}
\usepackage{graphicx}

\begin{document}
	\title{L"osung des Strahlungstransportproblems in Pfadintegralform mit effizienten Monte--Carlo--Verfahren}
	\author{Thies Heidecke\\(theidecke@astrophysik.uni-kiel.de)}
	\date{Version vom \today}
	\maketitle
	
	\pagebreak
	
	\section{Einleitung}
	\subsection{Motivation}
	\subsection{Ziele der Arbeit}
	\subsection{Nomenklatur}
	
	\section{Das Strahlungstransportproblem}
	TODO: Strahlungsgr"o"sen, Me"sgleichung, STG in diff. Form erkl"aren, "Ubersicht benutzter L"osungsverfahren
	
	\section{Pfadintegralformulierung der Strahlungstransportgleichung}
	TODO: STG in diff. Form, STG in integraler From, STG in Operator--Form, Messungen als Pfadintegrale
	\subsection{Strahlungstransport als Integrationsproblem}
	\section{Monte--Carlo--Integration}
	TODO: Begriffe aus Wahrscheinlichkeitsrechnung und Statistik einf"uhren, Das Integrationsproblem, Vergleich von Tensor--Produkt--Verfahren vs. MC--Integration (Curse~of~Dimensionality / Konvergenzraten), Importance~Sampling

	\subsection{Monte--Carlo--Markov--Chain--Verfahren}
	TODO: Metropolis--Hastings--Algorithmus, Detailed Balance, Robuste MH--Variante

	\section{Monte--Carlo--Strahlungstransport}
	TODO: Pfadgenerierung, Sch"atzer
	
	\section{Testf"alle}
	\subsection{Homogene streuende Kugel}
	\subsection{Einfaches Scheibenmodell}
	
	\section{Programmbeschreibung}
	\section{Resultate}
	\section{Zusammenfassung und Ausblick}
	\section{Literaturverzeichnis}

\end{document}