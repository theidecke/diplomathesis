\documentclass[11pt,a4paper,DIVcalc,BCOR8mm,titlepage,twoside]{scrartcl}

%\usepackage[parfill]{parskip}    % Activate to begin paragraphs with an empty line rather than an indent
\usepackage{german}
\usepackage{amsmath}
\usepackage{amsfonts}
\usepackage{amssymb}
\usepackage{graphicx}
\usepackage{natbib}
\bibpunct{(}{)}{;}{a}{}{,} % to follow the A&A style

\begin{document}
	\titlehead{Christian--Albrechts--Universit"at zu Kiel\\ Institut f"ur Theoretische Physik und Astrophysik}
	\subject{Diplomarbeit}
	\title{L"osung des Strahlungstransportproblems in Pfadintegralform mit effizienten Monte--Carlo--Verfahren}
	\author{vorgelegt von Thies Heidecke\\(theidecke@astrophysik.uni-kiel.de)}
	\publishers{betreut durch Prof. Sebastian Wolf}
	\date{Version vom \today}
	\maketitle
	

	\tableofcontents
	\pagebreak
	
	\newcommand{\location}[1]{\mathbf{#1}}
	\newcommand{\scatter}[1]{\overset{#1}{\leftrightsquigarrow}}
	\newcommand{\normalized}[1]{\frac{#1}{||#1||}}
	
	\section{Einleitung}
	\subsection{Motivation}
	\subsection{Ziele der Arbeit}
	In dieser Arbeit soll eine Pfadintegraldarstellung der Strahlungstransportgleichung hergeleitet werden, sowie dessen Nutzen f"ur Monte--Carlo--Verfahren zur L"osung des Strahlungstransportproblems theoretisch und praktisch in Form eines Computerprogrammes gezeigt werden.
	
	\subsection{Nomenklatur}
	Im folgenden Text werden die in Tabelle \ref{tab:nomenklatur} angegebenen Schreibweisen benutzt.

	\begin{table}
		\caption{Nomenklatur}
		\begin{center}
		\begin{tabular}{rll}
			Schreibweise & Bedeutung & Einheit \\
			\hline
			$\kappa$ & Volumenabsorptionsquerschnitt & $\left[\text{m}^2/\text{m}^3\right]$ \\
			$\sigma$ & Volumenstreuquerschnitt & $\left[\text{m}^2/\text{m}^3\right]$ \\
			$\xi$ & Volumenextinktionsquerschnitt & $\left[\text{m}^2/\text{m}^3\right]$ \\
			$\varepsilon$ & Volumenemissivit"at & $\left[\text{W}/(\text{m}^3\,\text{sr})\right]$ \\
			$I(\location{r},\omega)$ & Intensit"at am Ort $\location{r}$ in Richtung $\omega$& $\left[\text{W}/(\text{m}^2\,\text{sr})\right]$ \\
			$W(\location{r},\omega)$ & Sensitivit"at am Ort $\location{r}$ in Richtung $-\omega$ & $\left[(\text{m}^2\,\text{sr})/\text{W}\right]$ \\
			$\phi(\location{r},\omega',\omega)$ & Phasenfunktion am Ort $\location{r}$ f"ur ein Teilchen, & $\left[1/\text{sr}\right]$ \\
				&das sich vor der Streuung in Richtung $\omega'$&\\
				&und nach der Streuung in Richtung $\omega$ bewegt& \\
			$\scatter{i}$ & Phasenfunktion am Ort $\location{r}_i$ f"ur ein aus Richtung $\location{r}_{i-1}$&\\ 
				&kommendes und in Richtung $\location{r}_{i+1}$ gestreutes Teilchen&\\
				&("aquivalent zu $\phi(\location{r}_i,\normalized{\location{r}_i-\location{r}_{i-1}},\normalized{\location{r}_{i+1}-\location{r}_i})$)& \\
			$\tau_{i,j}$ & Optische Tiefe zwischen $\location{r}_i$ und $\location{r}_j$ & \\
			$\varepsilon_{i,j}$ & Emissivit"at am Ort $\location{r}_i$ in Richtung $\location{r}_j$ & \\
			$W_{j,i}$ & Sensitivit"at am Ort $\location{r}_i$ in Richtung $-\normalized{\location{r}_j-\location{r}_i}$ &
		\end{tabular}
		\end{center}
		\label{tab:nomenklatur}
	\end{table}
	
	
	\section{Das Strahlungstransportproblem}
	TODO: Strahlungsgr"o"sen, Me"sgleichung, STG in diff. Form erkl"aren, "Ubersicht benutzter L"osungsverfahren
	
	\section{Pfadintegralformulierung der Strahlungstransportgleichung}
	TODO: STG in diff. Form, STG in integraler From, STG in Operator--Form, Messungen als Pfadintegrale
	\subsection{Strahlungstransport als Integrationsproblem}
	\section{Monte--Carlo--Integration}
	TODO: Begriffe aus Wahrscheinlichkeitsrechnung und Statistik einf"uhren, Das Integrationsproblem, Vergleich von Tensor--Produkt--Verfahren vs. MC--Integration (Curse~of~Dimensionality / Konvergenzraten), Importance~Sampling

	\subsection{Monte--Carlo--Markov--Chain--Verfahren}
	TODO: Metropolis--Hastings--Algorithmus, Detailed Balance, Robuste MH--Variante

	\section{Monte--Carlo--Strahlungstransport}
	TODO: Pfadgenerierung, Sch"atzer
	
	\section{Testf"alle}
	\subsection{Homogene streuende Kugel}
	\subsection{Einfaches Scheibenmodell}
	
	\section{Programmbeschreibung}
	\section{Resultate}
	\section{Zusammenfassung und Ausblick}

	\bibliographystyle{chicago}
	\bibliography{bibliography}
\end{document}